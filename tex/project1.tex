\documentclass[11pt,a4paper,notitlepage]{article}
\usepackage[T1]{fontenc}
\usepackage[utf8]{inputenc}
\usepackage[english]{babel}
\usepackage{fullpage}
\usepackage{amsmath}
\usepackage{amsfonts}
\usepackage{amssymb}
\usepackage{verbatim}
\usepackage{listings}
\usepackage{color}
\usepackage{setspace}
\usepackage{epstopdf}
\usepackage{graphicx}
\usepackage{caption}
\usepackage{subcaption}
\usepackage{float}
\usepackage{epstopdf}
\usepackage{hyperref}
\usepackage{braket}
\pagenumbering{arabic}

\definecolor{codepurple}{rgb}{0.58,0,0.82}
\definecolor{backcolour}{rgb}{0.95,0.95,0.92}
\definecolor{dkgreen}{rgb}{0,0.6,0}
\definecolor{gray}{rgb}{0.5,0.5,0.5}
\definecolor{mauve}{rgb}{0.58,0,0.82}
%\setlength{\parindent}{0pt}

\lstdefinestyle{pystyle}{
  language=Python,
  aboveskip=3mm,
  belowskip=3mm,
  columns=flexible,
  basicstyle={\small\ttfamily},
  backgroundcolor=\color{backcolour},
  commentstyle=\color{dkgreen},
  keywordstyle=\color{magenta},
  numberstyle=\tiny\color{gray},
  stringstyle=\color{codepurple},
  basicstyle=\footnotesize,  
  breakatwhitespace=false
  breaklines=true,
  captionpos=b,
  keepspaces=true,
  numbers=left,
  numbersep=5pt,
  showspaces=false,
  showstringspaces=false,
  showtabs=false,
  tabsize=2
}
\lstdefinestyle{iStyle}{
  language=IDL,
  aboveskip=3mm,
  belowskip=3mm,
  columns=flexible,
  basicstyle={\small\ttfamily},
  backgroundcolor=\color{backcolour},
  commentstyle=\color{dkgreen},
  keywordstyle=\color{magenta},
  numberstyle=\tiny\color{gray},
  stringstyle=\color{codepurple},
  basicstyle=\footnotesize,  
  breakatwhitespace=false
  breaklines=true,
  captionpos=b,
  keepspaces=true,
  numbers=left,
  numbersep=5pt,
  showspaces=false,
  showstringspaces=false,
  showtabs=false,
  tabsize=2
}
	



\title{\normalsize Fys4150: Introduction to \\
\vspace{10mm}
\huge Project 1\\
\vspace{10mm}
\normalsize Due date {\bf 19.rd of September, 2016 - 23:59}}

% Skriv namnet ditt her og fjern kommenteringa
\author{Øyvind B. Svendsen, Magnus Christopher Bareid \\ un: oyvinbsv, magnucb}

\newcommand\pd[2]{\frac{\partial #1}{\partial #2}}
\def\doubleunderline#1{\underline{\underline{#1}}}


\begin{document}
\noindent
\maketitle
\vspace{10mm}
\begin{abstract}
The aim of this project is to get familiar with various vector and matrix operations, from dynamic memory allocation to the usage of programs in the library package of the course.

The student was invited to use either brute force-algorithms to calculate linear algebra, or to use a set of recommended linear algebra packages through Armadillo that simplify the syntax of linear algebra. Additionally, dynamic memory handling is expected.

The students will showcase necessary algebra to perform the tasks given to them, and explain the way said algebra is implemented into algorithms. In essence, we're asked to simplify a linear second-order differential equation from the form
\begin{align*}
\nabla ^2 \phi = -4\pi\rho(\mathbf{r})
\end{align*}
into a form akin to
\begin{align*}
-u''(x) = f(x)
\end{align*}
\end{abstract}
\begin{center}
\line(1,0){450}
\end{center}

\newpage
\tableofcontents

\newpage
\section{Computational Physics: First project}
\subsection{}
\subsubsection{}

\begin{center}
\line(1,0){450}
\end{center}

\newpage
\section{Appendix - Program list}
This is the code used in this assignment. Anything that was done by hand has been implemented into this pdf, above.
\lstset{style=pystyle}
%\subsection{General parameters document}
%\verb|solar_parameters.py|
%\lstinputlisting{solar_parameters.py}


\end{document}