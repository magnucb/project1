\documentclass[11pt,a4paper,notitlepage]{article}
\usepackage[T1]{fontenc}
\usepackage[utf8]{inputenc}
\usepackage[english]{babel}
\usepackage{fullpage}
\usepackage{amsmath}
\usepackage{amsfonts}
\usepackage{amssymb}
\usepackage{verbatim}
\usepackage{listings}
\usepackage{color}
\usepackage{setspace}
\usepackage{epstopdf}
\usepackage{graphicx}
\usepackage{caption}
\usepackage{subcaption}
\usepackage{float}
\usepackage{epstopdf}
\usepackage{hyperref}
\usepackage{braket}
\pagenumbering{arabic}

\definecolor{codepurple}{rgb}{0.58,0,0.82}
\definecolor{backcolour}{rgb}{0.95,0.95,0.92}
\definecolor{dkgreen}{rgb}{0,0.6,0}
\definecolor{gray}{rgb}{0.5,0.5,0.5}
\definecolor{mauve}{rgb}{0.58,0,0.82}
%\setlength{\parindent}{0pt}

\lstdefinestyle{pystyle}{
  language=Python,
  aboveskip=3mm,
  belowskip=3mm,
  columns=flexible,
  basicstyle={\small\ttfamily},
  backgroundcolor=\color{backcolour},
  commentstyle=\color{dkgreen},
  keywordstyle=\color{magenta},
  numberstyle=\tiny\color{gray},
  stringstyle=\color{codepurple},
  basicstyle=\footnotesize,  
  breakatwhitespace=false
  breaklines=true,
  captionpos=b,
  keepspaces=true,
  numbers=left,
  numbersep=5pt,
  showspaces=false,
  showstringspaces=false,
  showtabs=false,
  tabsize=2
}
\lstdefinestyle{iStyle}{
  language=IDL,
  aboveskip=3mm,
  belowskip=3mm,
  columns=flexible,
  basicstyle={\small\ttfamily},
  backgroundcolor=\color{backcolour},
  commentstyle=\color{dkgreen},
  keywordstyle=\color{magenta},
  numberstyle=\tiny\color{gray},
  stringstyle=\color{codepurple},
  basicstyle=\footnotesize,  
  breakatwhitespace=false
  breaklines=true,
  captionpos=b,
  keepspaces=true,
  numbers=left,
  numbersep=5pt,
  showspaces=false,
  showstringspaces=false,
  showtabs=false,
  tabsize=2
}
	



\title{\normalsize Fys4150: Introduction to \\
\vspace{10mm}
\huge Project 1\\
\vspace{10mm}
\normalsize Due date {\bf 19.rd of September, 2016 - 23:59}}

% Skriv namnet ditt her og fjern kommenteringa
\author{Øyvind B. Svendsen, Magnus Christopher Bareid \\ un: oyvinbsv, magnucb}

\newcommand\pd[2]{\frac{\partial #1}{\partial #2}}
\def\doubleunderline#1{\underline{\underline{#1}}}


\begin{document}
\noindent
\maketitle
\vspace{10mm}
\begin{abstract}
The aim of this project is to get familiar with various vector and matrix operations, from dynamic memory allocation to the usage of programs in the library package of the course.

The student was invited to use either brute force-algorithms to calculate linear algebra, or to use a set of recommended linear algebra packages through Armadillo that simplify the syntax of linear algebra. Additionally, dynamic memory handling is expected.

The students will showcase necessary algebra to perform the tasks given to them, and explain the way said algebra is implemented into algorithms. In essence, we're asked to simplify a linear second-order differential equation from the form of the Poisson equation, seen as
\begin{align*}
\nabla ^2 \Phi = -4\pi\rho(\mathbf{r})
\end{align*}
into a one-dimensional form bounded by Dirichlet boundary conditions.
\begin{align*}
-u''(x) = f(x)
\end{align*}
so that discretized linear algebra may be committed unto the equation.
\end{abstract}
\begin{center}
\line(1,0){450}
\end{center}

\newpage
\tableofcontents

\newpage
\section{Computational Physics: First project}
\subsection{(a): The fundamental math}
\subsubsection{Intro}
The production of this document will inevitably familiarize its authors with the programming language \verb|C++|, and to this end mathematical groundwork must first be elaborated to translate a Poisson equation from continuous calculus form, into a discretized numerical form.

The Poisson equation is rewritten to a simplified form, for which a real solution is given, with with we will compare our numerical approximation to the real solution.

\subsubsection{Method}
Reviewing the Poisson equation:
\begin{align*}
\nabla ^2 \Phi &= -4\pi\rho(\mathbf{r}), \ \text{which is simplified one-dimensionally by} \ \Phi(r) = \phi(r)/r \\
\Rightarrow \frac{d^2 \phi}{dr^2} &= -4\pi r \rho(r), \ \text{which is further simplified by these substitutions:}\\
r &\rightarrow x, \\
\phi &\rightarrow u,\\
4\pi r\rho(r) &\rightarrow f, \indent \text{which produces the simplified form}
\end{align*}
\begin{align*}\label{eq:1}\tag{1}
-u''(x) &= f(x), \indent \text{for which we assume that} \indent f(x) = 100e^{-10x}, \\
\Rightarrow u(x) &= 1- (1-e^{-10})x - e^{-10x},\ \text{with bounds:}\ x \in [0,1],\ u(0) = u(1)= 0
\end{align*}
To more easily comprehend the syntax from a programming viewpoint, one may refer to the each discretized representation of $x$ and $u$; we know the span of $x$, and therefore we may divide it up into appropriate chunks. Each of these $x_i$ will yield a corresponding $u_i$.

We may calculate each to each discrete $x_i$ by the form $x_i = ih$ in the interval from $x_0 = 0$ to $x_{n} = 1$ as it is linearly increasing, meaning we use $n$ points in our approximation, yielding the step length $h = 1/n$. Of course, this also yields for the discretized representation of $u(x_i) = u_i$.


Through Euler's teachings on discretized numerical derivation methods, a second derivative may be constructed through the form of
\begin{align}\label{eq:2}\tag{2}
-u''(x) = - \frac{u_{+1} + u_{i-1} - 2u_i}{h^2} = \frac{2u_i - u_{+1} - u_{i-1}}{h^2} = f_i , \indent \text{for}\ i = 1,...,n
\end{align}
\subsubsection{Results}
\subsubsection{Discussion}

\subsection{(b)}
\subsubsection{Intro}
\subsubsection{Method}
\subsubsection{Results}
\subsubsection{Discussion}

\subsection{(c)}
\subsubsection{Intro}
\subsubsection{Method}
\subsubsection{Results}
\subsubsection{Discussion}

\subsection{(d)}
\subsubsection{Intro}
\subsubsection{Method}
\subsubsection{Results}
\subsubsection{Discussion}

\subsection{(e)}
\subsubsection{Intro}
\subsubsection{Method}
\subsubsection{Results}
\subsubsection{Discussion}


\begin{center}
\line(1,0){450}
\end{center}

\newpage
\section{Appendix - Program list}
This is the code used in this assignment. Anything that was done by hand has been implemented into this pdf, above.
\lstset{style=pystyle}
\verb|plot_stuff.py|
\lstinputlisting{../plot_stuff.py}

\end{document}
